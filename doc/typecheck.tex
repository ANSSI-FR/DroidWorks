\documentclass[english,dvips,ps2pdf,11pt]{article}

\usepackage[utf8]{inputenc}
\usepackage[T1]{fontenc}
\usepackage{babel}
\usepackage{microtype}

\usepackage[top=2cm,bottom=2cm,right=2cm,left=2cm]{geometry}

\usepackage{xspace}
\usepackage{amsfonts}
\usepackage{amsmath}
\usepackage{amssymb}
\usepackage{amsthm}
\usepackage{bussproofs}
\usepackage{color}
\usepackage{hyperref}
\usepackage{pst-node}

\def\defaultHypSeparation{\hspace*{.5em}}

\let\powerset\wp
\theoremstyle{definition}
\newtheorem{definition}{Definition}[section]


\newcommand{\todo}[1]{\footnote{\textcolor{red}{#1}}}

\newcommand{\var}[1]{\ensuremath{\mathit{#1}}\xspace}
\newcommand{\func}[1]{\ensuremath{\mathsf{#1}}\xspace}

\newcommand{\bcode}[1]{\texttt{#1}\xspace}

\newcommand{\dalvik}[0]{\textsc{Dalvik}\xspace}

\newcommand{\databool}[0]{\ensuremath{\texttt{bool}}\xspace}
\newcommand{\datainteger}[0]{\ensuremath{\texttt{integer}}\xspace}
\newcommand{\datafloat}[0]{\ensuremath{\texttt{float}}\xspace}
\newcommand{\datalong}[0]{\ensuremath{\texttt{long}}\xspace}
\newcommand{\datadouble}[0]{\ensuremath{\texttt{double}}\xspace}
\newcommand{\meettd}[0]{\ensuremath{\textsf{Meet32}}\xspace}
\newcommand{\meetsq}[0]{\ensuremath{\textsf{Meet64}}\xspace}
\newcommand{\jointd}[0]{\ensuremath{\textsf{Join32}}\xspace}
\newcommand{\joinsq}[0]{\ensuremath{\textsf{Join64}}\xspace}
\newcommand{\joinz}[0]{\ensuremath{\textsf{JoinZero}}\xspace}
%\newcommand{\refarray}[2]{\ensuremath{\textsf{Array}^{#1}_ {#2}}\xspace}
\newcommand{\refarray}[2]{\ensuremath{[#2]^{#1}}\xspace}
%\newcommand{\refobject}[1]{\ensuremath{\textsf{Object}_{#1}}\xspace}
\newcommand{\refobject}[1]{\ensuremath{\langle #1 \rangle}\xspace}
\newcommand{\meetz}[0]{\ensuremath{\textsf{MeetZero}}\xspace}
\newcommand{\pnull}[0]{\ensuremath{\textsf{Null}}\xspace}
\newcommand{\valuetypes}[0]{\ensuremath{\mathcal{T_J}}\xspace}
\newcommand{\classtypes}[0]{\ensuremath{\mathcal{T_C}}\xspace}
\newcommand{\concretetypes}[0]{\ensuremath{\mathcal{T}}\xspace}
\newcommand{\types}[0]{\ensuremath{\overline{\mathcal{T}}}\xspace}
\newcommand{\classes}[0]{\ensuremath{\mathcal{C}}\xspace}
\newcommand{\fields}[0]{\ensuremath{\mathcal{F}}\xspace}
\newcommand{\methods}[0]{\ensuremath{\mathcal{M}}\xspace}
\newcommand{\jumps}[0]{\ensuremath{\mathcal{L}}\xspace}
\newcommand{\instructions}[0]{\ensuremath{\mathcal{I}}\xspace}
\newcommand{\instruction}[1]{\ensuremath{\var{instr}_{#1}}\xspace}
\newcommand{\cfg}[1]{\var{CFG}_{#1}}

\newcommand{\pfields}[0]{\ensuremath{\mathcal{A}_\var{fields}}\xspace}
\newcommand{\pstrings}[0]{\ensuremath{\mathcal{A}_\var{strings}}\xspace}
\newcommand{\ptypes}[0]{\ensuremath{\mathcal{A}_\var{types}}\xspace}
\newcommand{\pmethods}[0]{\ensuremath{\mathcal{A}_\var{methods}}\xspace}

\newcommand{\aclass}[1]{\texttt{#1}\xspace}

\newcommand{\jobject}[0]{\ensuremath{\texttt{java.lang.Object}}\xspace}
\newcommand{\jclass}[0]{\ensuremath{\texttt{java.lang.Class}}\xspace}
\newcommand{\jstring}[0]{\ensuremath{\texttt{java.lang.String}}\xspace}
\newcommand{\jexception}[0]{\ensuremath{\texttt{java.lang.Exception}}\xspace}
\newcommand{\jserializable}[0]{\ensuremath{\texttt{java.io.Serializable}}\xspace}

\newcommand{\resolve}[0]{\var{resolve}}
\newcommand{\typeof}[0]{\ensuremath{\tau}\xspace}
%\newcommand{\parentof}[0]{\ensuremath{\upsilon}\xspace}
\newcommand{\inherits}[0]{\ensuremath{\leq}\xspace}
\newcommand{\cinherits}[0]{\ensuremath{\sqsubseteq}\xspace}
\newcommand{\cjoin}[0]{\ensuremath{\sqcup}\xspace}

\newcommand{\nbregisters}[0]{\func{nbRegisters}}
\newcommand{\nbparams}[0]{\func{nbParams}}
\newcommand{\returnof}[0]{\func{return}}
\newcommand{\parameter}[0]{\func{parameter}}
\newcommand{\static}[0]{\func{static}}
\newcommand{\switchoffsets}[0]{\func{switchOffsets}}
\newcommand{\constructor}[0]{\func{constructor}}
\newcommand{\final}[0]{\func{final}}
\newcommand{\private}[0]{\func{private}}
\newcommand{\interface}[0]{\func{interface}}
\newcommand{\accessible}[0]{\func{accessible}}
\newcommand{\checkaccess}[0]{\func{checkAccess}}

\newcommand{\rvoid}[0]{\ensuremath{\epsilon}\xspace}

\newcommand{\registers}[0]{\ensuremath{\rho}\xspace}
\newcommand{\lastresult}[0]{\ensuremath{\lambda}\xspace}


%\newcommand{\subtype}[0]{\ensuremath{\preceq}\xspace}
\newcommand{\subtype}[0]{\ensuremath{<:}\xspace}

\newenvironment{sequent}[0]{\begin{prooftree}}{\end{prooftree}}

\newcommand{\ie}[0]{\textit{i.e.}\xspace}



\newcommand{\reffig}[1]{Figure~\ref{#1}\xspace}



\title{Dalvik bytecode verification}
\author{Arnaud Fontaine}



\begin{document}

\maketitle

\paragraph{Prerequisites}

\begin{definition}[Application package]
  \todo{revoir cette définition foireuse}
  An application, \ie an APK in Android's terminology,
  contains a \texttt{classes.dex} with several pools:
  \begin{itemize}
  \item \pstrings : a set of string constants;
  \item \ptypes : a set of type descriptors;
  \item \pfields : a set of (class) field descriptors;
  \item \pmethods : a set of method descriptors.
  \end{itemize}
\end{definition}

\begin{definition}[Dalvik VM]
  \begin{itemize}
  \item \classes : \dots %see below \todo{définir que les trucs suivants existent dans \classes : $\jobject = \aclass{java.lang.Object}$, $\jstring = \aclass{java.lang.String}$, $\jclass = \aclass{java.lang.Class}$, $\jexception = \aclass{java.lang.Exception}$}
  \item $\fields = \left\{ c.f \mid c \in \classes \right\} $
  \item $\methods = \left\{ c.m \mid c \in \classes \right\} $
  \end{itemize}
\end{definition}

\begin{definition}[Join-semilattice of class hierarchy]
  %% Join-semilattice $(\classes, \leq, \vee)$ of class hierarchy, where $C_1 \leq C_2$ means 
  %% $C_1$ inherits (or implements) $C_2$. \todo{definir $\vee$}
  $\classes$ is the set of all classes, interfaces.

  $(\inherits) : \classes \times \classes \longrightarrow \mathbb{B}$, means inherits or implements.

  $\classtypes \triangleq \powerset_{\geq 1}(\classes)$ \todo{attention, à cause de la relation d'héritage à la Java, ce n'est pas un semi-lattice.. voir page 6 et 9 (papier xavier). On exprime ici la conjonction de types auxquels appartient une instance.}

  \begin{gather*}
  \begin{aligned}
    (\cinherits) : \classtypes \times \classtypes & \longrightarrow \classtypes\\
    C_1 \cinherits C_2 & \equiv \forall c_2 \in C_2 \exists c_1 \in C_1\ c_1 \inherits c_2\\
  \end{aligned}\\
  \begin{aligned}
    (\cjoin) : \classtypes \times \classtypes & \longrightarrow \classtypes\\
    C_1 \cjoin C_2 & \triangleq \begin{cases}
      C_2 & \text{if $C_1$ \cinherits $C_2$}\\
      C_1 & \text{if $C_2$ \cinherits $C_1$}\\
      \left\{ c \mid \exists (c_1, c_2) \in C_1 \times C_2, c_1 \inherits c, c_2 \inherits c, \nexists c' \in \classes\ (c' \neq c) \wedge (c_1 \inherits c' \inherits c) \wedge (c_2 \inherits c' \inherits c) \right\} & \text{otherwise}\\
      \end{cases}
  \end{aligned}
  \end{gather*}

  Join-semilattice $(\classtypes, \cinherits, \cjoin)$
  \todo{montrer qu'il existe toujours un plus grand élément, $\{\jobject\}$}

\end{definition}

\begin{definition}[Concrete types]
  Value types $\valuetypes \triangleq \left\{ \datainteger, \datafloat, \datalong, \datadouble \right\} \cup \left\{ \refobject{c_1, \dots, c_n} \mid \{c_1,\dots, c_n\} \in \classtypes \right\}$.

  Concrete types $\concretetypes \triangleq \valuetypes \cup \left\{ \refarray{p}{t} \mid p \in \mathbb{N}^*, t \in \valuetypes \right\}$.
\end{definition}


\begin{definition}[Join-semilattice of types]
  Join-semilattice $(\types, \leq, \vee)$ of types\todo{\types est potentiellement infini à cause de l'utilisation d'un entier pour coder la taille d'un tableau, il faudra qu'il soit fini pour le point fixe !!!} with
  \[
  \types \triangleq \concretetypes \cup \left\{ \top, \pnull, \meetz, \meettd, \meetsq, \joinz, \jointd, \joinsq \right\}
  \]
  
  The ordering relation $\leq$ and the join operation $\vee$ are defined by
  \begin{center}

    \begin{tabular}{cccccccc}
      &&&& \rnode{top}{$\top$}\\[2em]
      & \rnode{j64}{\joinsq} &&& \rnode{j32}{\jointd} & \rnode{jzero}{\joinz}\\[2em]
      &&&&&& \rnode{ref}{$\refobject{\jobject}$}\\[1em]
      \rnode{double}{\datadouble} & \rnode{long}{\datalong} &&& \rnode{float}{\datafloat} & \rnode{integer}{\datainteger} & \rnode{refo}{$\refobject{c_1,\dots,c_n}$} & \rnode{refa}{$\refarray{p}{t}$}\\[2em]
      & \rnode{m64}{\meetsq} &&&& \rnode{m32}{\meettd} & \rnode{null}{\pnull}\\[2em]
      &&&&& \rnode{mzero}{\meetz}\\[2em]
      &&&& \rnode{bot}{$\bot$}
    \end{tabular}

    %%%%

    \psset{nodesep=4pt}

    \ncline{-}{bot}{m64}
    \ncline{-}{bot}{mzero}

    \ncline{-}{m64}{long}
    \ncline{-}{m64}{double}

    \ncline{-}{mzero}{m32}
    \ncline{-}{mzero}{null}

    \ncline{-}{refo}{ref}
    \ncline{-}{refa}{ref}
    
    \ncline{-}{null}{refo}
    \ncline{-}{null}{refa}
    
    \ncline{-}{m64}{long}
    \ncline{-}{m64}{double}

    \ncline{-}{m32}{integer}
    \ncline{-}{m32}{float}
    
    \ncline{-}{long}{j64}
    \ncline{-}{double}{j64}
    \ncline{-}{integer}{j32}
    \ncline{-}{float}{j32}
    
    \ncline{-}{integer}{jzero}
    \ncline{-}{ref}{jzero}
    
    \ncline{-}{j64}{top}
    \ncline{-}{j32}{top}
    \ncline{-}{jzero}{top}
  \end{center}
  with
  $\refobject{C_1} \leq \refobject{C_2} \equiv C_1 \cinherits C_2$,
  $\refobject{C_1} \vee \refobject{C_2} \equiv \refobject{C_1 \cjoin C_2}$,
  \[
  \refarray{p_1}{t_1} \leq \refarray{p_2}{t_2} \equiv \begin{cases}
    C_1 \cinherits C_2 & \text{if $p_1 = p_2$ and $t_1 = \refobject{C_1}$ and $t_2 = \refobject{C_2}$}\\
    \var{true} & \text{if ($p_1 = p_2$ and $t_1 = t_2$) or ($p_2 < p_1$ and $t_2 = \refobject{\jobject}$)}\\
    \var{false} & \text{otherwise}\\
    \end{cases}
  \]
  and
  \[
  \refarray{p_1}{t_1} \vee \refarray{p_2}{t_2} \triangleq \begin{cases}
    \refarray{p_2}{t_2} & \text{if $\refarray{p_1}{t_1} \leq \refarray{p_2}{t_2}$}\\ 
    \refarray{p_1}{t_1} & \text{if $\refarray{p_2}{t_2} \leq \refarray{p_1}{t_1}$}\\
    \refarray{p_1}{\refobject{C_1 \cjoin C_2}} & \text{if $p_1 = p_2$ and $t_1 = \refobject{C_1}$ and $t_2 = \refobject{C_2}$}\\
    \refobject{\jobject} & \text{otherwise}\\
  \end{cases}
  \]
\end{definition}


\begin{definition}[Valide subtype relation \subtype]
  $t_1 \subtype t_2 \equiv t_1 \leq t_2 \wedge t_2 \neq \top$
\end{definition}



\begin{definition}[Function \resolve]
  \begin{align*}
    \resolve & : \ptypes \longrightarrow \concretetypes\\
    \resolve & : \pfields \longrightarrow \fields\\
    \resolve & : \pmethods \longrightarrow \methods\\
  \end{align*}
\end{definition}

\begin{definition}[Function \typeof]
  $\typeof : \fields \longrightarrow \concretetypes$
\end{definition}

\begin{definition}[Function \nbparams]
  $\nbparams : \methods \longrightarrow \mathbb{N}$
\end{definition}

\begin{definition}[Function \returnof]
  $\returnof : \methods \longrightarrow \concretetypes \cup \{\rvoid\}$
\end{definition}

\begin{definition}[Function \parameter]
  $\parameter : \methods \times \mathbb{N} \longrightarrow \concretetypes$
\end{definition}

\begin{definition}[Function \static]
  $\static : \fields \cup \methods \longrightarrow \mathbb{B}$
\end{definition}

\begin{definition}[Function \switchoffsets]
  $\switchoffsets : \mathbb{N} \times \mathbb{N} \longrightarrow \powerset(\mathbb{N})$
\end{definition}

\begin{definition}[Function \constructor]
  $\constructor : \methods \longrightarrow \mathbb{B}$
\end{definition}

\begin{definition}[Function \interface]
  $\interface : \classes \longrightarrow \mathbb{B}$
\end{definition}

\begin{definition}[Function \accessible]
  $\accessible : \classes \times (\fields \cup \methods) \longrightarrow \mathbb{B}$
\end{definition}




The intraprocedural control flow graph of a method can be built
linearly since overlapping sequences of instructions is not permitted
in \dalvik's bytecode. The sequence of instructions of $c.m$ is thus
sequentially decoded from the first byte of the related code data to
build the mapping $\instruction{c.m} : \mathbb{N} \longrightarrow
\instructions$ such that $\instruction{c.m}(\var{pc})$ is defined iff
\var{pc} is an offset to such a decoded instruction; otherwise,
$\instruction{c.m}(\var{pc})$ is undefined.\todo{définir que $|b|$ donne la taille de l'instruction décodée} \todo{préciser que certaines instructions embarquent des données, immédiatement (fill array) ou plus loin (switch)}

\begin{definition}[Intraprocedural control flow graph]
  The intraprocedural control flow graph of a method $c.m \in \methods$
  is a graph $CFG_{c.m} \triangleq (V_{c.m}, E_{c.m})$ such that
  \begin{align*}
    V_{c.m} \triangleq \{ & (\var{pc}, \instruction{c.m}(\var{pc})) \mid \var{pc} \in \var{dom}(\instruction{c.m})\}\\
    E_{c.m} \triangleq \{ & (v_1, v_2, j) \mid v_1 \in V_{c.m}, v_2 \in V_{c.m}, j \in \jumps, v_1 = (\var{pc}_1, b_1), v_2 = (\var{pc}_2, b_2),\\
    &\quad\quad\quad\quad\wedge (b_1 = \bcode{if-*}\ r_1\ r_2\ n \implies (\var{pc}_2 = \var{pc}_1 + n \wedge j = ) \vee (\var{pc}_2 = \var{pc}_1 + |b_1|)) \\
    &\quad\quad\quad\quad\wedge (b_1 = \bcode{goto}\ n \implies \var{pc}_2 = \var{pc}_1 + n)\\
    &\quad\quad\quad\quad\wedge (b_1 = \bcode{*-switch}\ r\ n\ \implies \var{pc}_2 \in \switchoffsets(\var{pc}_1, n))\\
    &\quad\quad\quad\quad\wedge (b_1 \notin \{ \bcode{return-*}\dots, \bcode{if-*}\dots, \bcode{goto}\dots, \bcode{*-switch}\dots, \bcode{throw}\dots \} \implies \var{pc}_2 = \var{pc}_1 + |b_1|) \}\\
  \end{align*}
\end{definition}




\todo{TRES IMPORTANT : justifier d'une première étape où l'on aura vérifié les paramètres fields, methods, strings et types pour ne plus avoir que les notres !!! \ie homomorphisme en appliquant \resolve sur les parametres nécessaires des opcodes}



\begin{definition}[Program state]
  
  For a method $c.m \in \methods$ and $\var{pc} \in \var{dom}(\instruction{c.m})$

  $Q_{c.m}^\var{pc} \triangleq (\registers_{c.m}^\var{pc}, \lastresult_{c.m}^\var{pc})$
  with
  $\registers_{c.m}^\var{pc}: \{ i \mid 0 \leq i < \nbregisters(c.m)\} \longrightarrow \types$ and $\lastresult_{c.m}^\var{pc} \in \types \cup \{\rvoid\}$.

  Initial state
  $\forall c.m \in \methods,\ Q_{c.m}^0 \triangleq (\registers_{c.m}^0, \rvoid)$ with
  $\forall r \in \var{dom}(\registers_{c.m})\ \registers_{c.m}^0(r) \triangleq \top$.

  
\end{definition}




\begin{definition}[Function \checkaccess]
  $\checkaccess : \methods \times \instructions  \longrightarrow \mathbb{B}$

  \[\scriptsize
  \begin{array}{l}
    \checkaccess(c.m, b) =\\
    \begin{cases}
      \accessible(c, c'.f) \wedge \neg\static(c'.f) & \text{if $b = \bcode{iget}*\ r_d\ r_o\ c'.f$}\\
      \accessible(c, c'.f) \wedge \neg\static(c'.f) \wedge (\final(c'.f) \implies \constructor(c.m)) & \text{if $b = \bcode{iput}*\ r_s\ r_o\ c'.f$}\\
      \accessible(c, c'.f) \wedge \static(c'.f) & \text{if $b = \bcode{sget}*\ r_d\ r_o\ c'.f$}\\
      \accessible(c, c'.f) \wedge \static(c'.f) \wedge \neg\final(c'.f)\todo{euh, comment ils sont initialisés ? comme en java classique ?} & \text{if $b = \bcode{sput}*\ r_s\ c'.f$}\\
      \accessible(c, c'.m') \wedge \neg\static(c'.m') \wedge \neg\interface(c') \wedge (\constructor(c'.m') \vee \private(c'.m') \vee \final(c'.m')) & \text{if $b = \bcode{invoke-direct}\ r_i\ r_1 \dots r_n\ c'.m'$}\\
      \accessible(c, c'.m') \wedge \neg\static(c'.m') \wedge \interface(c') \wedge \neg\constructor(c'.m')\todo{on pourrait ajouter $\neg(\constructor(c'.m') \vee \private(c'.m') \vee \final(c'.m'))$ mais normalement c'est inutile car on aura vérifié l'interface $c'$ ne définit pas de private et/ou de final et/ou de constructeur} & \text{if $b = \bcode{invoke-interface}\ r_i\ r_1 \dots r_n\ c'.m'$}\\
      \accessible(c, c'.m') \wedge \static(c'.m') \wedge \neg\interface(c')\todo{on aura vérifié qu'une méthode static n'est pas un constructeur, ou pas ???} & \text{if $b = \bcode{invoke-static}\ r_1 \dots r_n\ c'.m'$}\\
      \accessible(c, c'.m') \wedge \neg\static(c'.m') \wedge \neg\interface(c')  \wedge c \inherits c' \wedge (\constructor(c'.m') \implies \constructor(c.m)) & \text{if $b = \bcode{invoke-super}\ r_i\ r_1 \dots r_n\ c'.m'$}\\
      \accessible(c, c'.m') \wedge \neg(\static(c'.m') \vee \interface(c') \vee \private(c'.m') \vee \constructor(c'.m')) &\text{if $b =  \bcode{invoke-virtual}\ r_i\ r_1 \dots r_n\ c'.m'$}\\
      \var{true} & \text{otherwise}\\
    \end{cases}\\
  \end{array}
  \]
  
\end{definition}








\appendix

\section{Typing rules}
%The following statement holds for all following transformation rules:
\[Q = (\registers, \lastresult)\]

The method currently analysed is $c.m \in \methods$.

{\scriptsize

\begin{sequent}
  \AxiomC{$b = \bcode{aget}\ r_d\ r_a\ r_i$}
  \AxiomC{$\registers(r_a) \subtype \pnull$}
  \AxiomC{$\registers(r_i) \subtype \dataint$}
  \TrinaryInfC{$Q' = (\registers[r_d \mapsto \meettd], \rvoid)$}
\end{sequent}

\begin{sequent}
  \AxiomC{$b = \bcode{aget}\ r_d\ r_a\ r_i$}
  \AxiomC{$\registers(r_a) = \refarray{1}{t}$ \defaultHypSeparation $t \in \{ \dataint, \datafloat \}$}
  \AxiomC{$\registers(r_i) \subtype \dataint$}
  \TrinaryInfC{$Q' = (\registers[r_d \mapsto t], \rvoid)$}
\end{sequent}

\begin{sequent}
  \AxiomC{$b = \bcode{aget-boolean}\ r_d\ r_a\ r_i$}
  \AxiomC{$\registers(r_a) \subtype \pnull \vee \registers(r_a) = \refarray{1}{\databool}$}
  \AxiomC{$\registers(r_i) \subtype \dataint$}
  \TrinaryInfC{$Q' = (\registers[r_d \mapsto \databool], \rvoid)$}
\end{sequent}

\begin{sequent}
  \AxiomC{$b = \bcode{aget-byte}\ r_d\ r_a\ r_i$}
  \AxiomC{$\registers(r_a) \subtype \pnull \vee \registers(r_a) = \refarray{1}{\databyte}$}
  \AxiomC{$\registers(r_i) \subtype \dataint$}
  \TrinaryInfC{$Q' = (\registers[r_d \mapsto \databyte], \rvoid)$}
\end{sequent}

\begin{sequent}
  \AxiomC{$b = \bcode{aget-char}\ r_d\ r_a\ r_i$}
  \AxiomC{$\registers(r_a) \subtype \pnull \vee \registers(r_a) = \refarray{1}{\datachar}$}
  \AxiomC{$\registers(r_i) \subtype \dataint$}
  \TrinaryInfC{$Q' = (\registers[r_d \mapsto \datachar], \rvoid)$}
\end{sequent}

\begin{sequent}
  \AxiomC{$b = \bcode{aget-object}\ r_d\ r_a\ r_i$}
  \AxiomC{$\registers(r_a) \subtype \pnull$}
  \AxiomC{$\registers(r_i) \subtype \dataint$}
  \TrinaryInfC{$Q' = (\registers[r_d \mapsto \pnull], \rvoid)$}
\end{sequent}

\begin{sequent}
  \AxiomC{$b = \bcode{aget-object}\ r_d\ r_a\ r_i$}
  \AxiomC{$\registers(r_a) = \refarray{p}{t}$ \defaultHypSeparation $p > 1$}
  \AxiomC{$\registers(r_i) \subtype \dataint$}
  \TrinaryInfC{$Q' = (\registers[r_d \mapsto \refarray{p - 1}{t}], \rvoid)$}
\end{sequent}

\begin{sequent}
  \AxiomC{$b = \bcode{aget-object}\ r_d\ r_a\ r_i$}
  \AxiomC{$\registers(r_a) = \refarray{1}{t}$ \defaultHypSeparation $t \subtype \refobject{\jobject}$}
  \AxiomC{$\registers(r_i) \subtype \dataint$}
  \TrinaryInfC{$Q' = (\registers[r_d \mapsto t], \rvoid)$}
\end{sequent}

\begin{sequent}
  \AxiomC{$b = \bcode{aget-short}\ r_d\ r_a\ r_i$}
  \AxiomC{$\registers(r_a) \subtype \pnull \vee \registers(r_a) = \refarray{1}{\datashort}$}
  \AxiomC{$\registers(r_i) \subtype \dataint$}
  \TrinaryInfC{$Q' = (\registers[r_d \mapsto \datashort], \rvoid)$}
\end{sequent}

\begin{sequent}
  \AxiomC{$b = \bcode{aget-wide}\ r_d\ r_a\ r_i$}
  \AxiomC{$\registers(r_a) \subtype \pnull$}
  \AxiomC{$\registers(r_i) \subtype \dataint$}
  \TrinaryInfC{$Q' = (\registers[r_d \mapsto \meetsq, r_d + 1 \mapsto \meetsq], \rvoid)$}
\end{sequent}

\begin{sequent}
  \AxiomC{$b = \bcode{aget-wide}\ r_d\ r_a\ r_i$}
  \AxiomC{$\registers(r_a) = \refarray{1}{t}$ \defaultHypSeparation $t \subtype \joinsq$}
  \AxiomC{$\registers(r_i) \subtype \dataint$}
  \TrinaryInfC{$Q' = (\registers[r_d \mapsto t, r_d + 1 \mapsto t], \rvoid)$}
\end{sequent}

\begin{sequent}
  \AxiomC{$b = \bcode{aput}\ r_s\ r_a\ r_i$}
  \AxiomC{$\registers(r_a) \subtype \pnull \vee \registers(r_a) = \refarray{1}{t}$ \defaultHypSeparation $\registers(r_s) \subtype \jointd$}
  \AxiomC{$\registers(r_i) \subtype \dataint$}
  \TrinaryInfC{$Q' = (\registers, \rvoid)$}
\end{sequent}

\begin{sequent}
  \AxiomC{$b = \bcode{aput-boolean}\ r_s\ r_a\ r_i$}
  \AxiomC{$\registers(r_a) \subtype \pnull \vee \registers(r_a) = \refarray{1}{\databool}$ \defaultHypSeparation $\registers(r_s) \subtype \databool$}
  \AxiomC{$\registers(r_i) \subtype \dataint$}
  \TrinaryInfC{$Q' = (\registers, \rvoid)$}
\end{sequent}

\begin{sequent}
  \AxiomC{$b = \bcode{aput-byte}\ r_s\ r_a\ r_i$}
  \AxiomC{$\registers(r_a) \subtype \pnull \vee \registers(r_a) = \refarray{1}{\databyte}$ \defaultHypSeparation $\registers(r_s) \subtype \databyte$}
  \AxiomC{$\registers(r_i) \subtype \dataint$}
  \TrinaryInfC{$Q' = (\registers, \rvoid)$}
\end{sequent}

\begin{sequent}
  \AxiomC{$b = \bcode{aput-char}\ r_s\ r_a\ r_i$}
  \AxiomC{$\registers(r_a) \subtype \pnull \vee \registers(r_a) = \refarray{1}{\datachar}$ \defaultHypSeparation $\registers(r_s) \subtype \datachar$}
  \AxiomC{$\registers(r_i) \subtype \dataint$}
  \TrinaryInfC{$Q' = (\registers, \rvoid)$}
\end{sequent}

\begin{sequent}
  \AxiomC{$b = \bcode{aput-object}\ r_s\ r_a\ r_i$}
  \AxiomC{$\registers(r_a) \subtype \pnull$ \defaultHypSeparation $\registers(r_s) \subtype \refobject{\jobject}$}
  \AxiomC{$\registers(r_i) \subtype \dataint$}
  \TrinaryInfC{$Q' = (\registers, \rvoid)$}
\end{sequent}

\begin{sequent}
  \AxiomC{$b = \bcode{aput-object}\ r_s\ r_a\ r_i$}
  \AxiomC{$\registers(r_a) = \refarray{p}{t}$ \defaultHypSeparation $p > 1$ \defaultHypSeparation $\registers(r_s) \subtype \refarray{p - 1}{t}$}
  \AxiomC{$\registers(r_i) \subtype \dataint$}
  \TrinaryInfC{$Q' = (\registers, \rvoid)$}
\end{sequent}

\begin{sequent}
  \AxiomC{$b = \bcode{aput-object}\ r_s\ r_a\ r_i$}
  \AxiomC{$\registers(r_a) = \refarray{1}{t}$ \defaultHypSeparation $\registers(r_s) \subtype t \subtype \refobject{\jobject}$}
  \AxiomC{$\registers(r_i) \subtype \dataint$}
  \TrinaryInfC{$Q' = (\registers, \rvoid)$}
\end{sequent}

\begin{sequent}
  \AxiomC{$b = \bcode{aput-short}\ r_s\ r_a\ r_i$}
  \AxiomC{$\registers(r_a) \subtype \pnull \vee \registers(r_a) = \refarray{1}{\datashort}$ \defaultHypSeparation $\registers(r_s) \subtype \datashort$}
  \AxiomC{$\registers(r_i) \subtype \dataint$}
  \TrinaryInfC{$Q' = (\registers, \rvoid)$}
\end{sequent}

\begin{sequent}
  \AxiomC{$b = \bcode{aput-wide}\ r_s\ r_a\ r_i$}
  \AxiomC{$\registers(r_a) \subtype \pnull$ \defaultHypSeparation $\registers(r_s) \subtype \joinsq$ \defaultHypSeparation $\registers(r_s) = \registers(r_s + 1)$}
  \AxiomC{$\registers(r_i) \subtype \dataint$}
  \TrinaryInfC{$Q' = (\registers, \rvoid)$}
\end{sequent}

\begin{sequent}
  \AxiomC{$b = \bcode{aput-wide}\ r_s\ r_a\ r_i$}
  \AxiomC{$\registers(r_a) = \refarray{1}{t}$ \defaultHypSeparation $\registers(r_s) \subtype t \subtype \joinsq$ \defaultHypSeparation $\registers(r_s) = \registers(r_s + 1)$}
  \AxiomC{$\registers(r_i) \subtype \dataint$}
  \TrinaryInfC{$Q' = (\registers, \rvoid)$}
\end{sequent}

\begin{sequent}
  \AxiomC{$b = \bcode{array-length}\ r_d\ r_a$}
  \AxiomC{$\registers(r_a) \subtype \pnull \vee \registers(r_a) = \refarray{p}{t}$}
  \BinaryInfC{$Q' = (\registers[r_d \mapsto \dataint], \rvoid)$}
\end{sequent}

\begin{sequent}
  \AxiomC{$b = \var{binop}\bcode{-double}\ r_d\ r_{s_1}\ r_{s_2}$}
  \AxiomC{$\registers(r_{s_1}) \subtype \datadouble$ \defaultHypSeparation $\registers(r_{s_1}) = \registers(r_{s_1}+1)$}
  \AxiomC{$\registers(r_{s_2}) \subtype \datadouble$ \defaultHypSeparation $\registers(r_{s_2}) = \registers(r_{s_2}+1)$}
  \TrinaryInfC{$Q' = (\registers[r_d \mapsto \datadouble, r_d + 1\mapsto \datadouble], \rvoid)$}
\end{sequent}

\begin{sequent}
  \AxiomC{$b = \var{binop}\bcode{-double/2addr}\ r_d\ r_s$}
  \AxiomC{$\registers(r_d) \subtype \datadouble$ \defaultHypSeparation $\registers(r_d) = \registers(r_d + 1)$}
  \AxiomC{$\registers(r_s) \subtype \datadouble$ \defaultHypSeparation $\registers(r_s) = \registers(r_s + 1)$}
  \TrinaryInfC{$Q' = (\registers[r_d \mapsto \datadouble, r_d + 1\mapsto \datadouble], \rvoid)$}
\end{sequent}

\begin{sequent}
  \AxiomC{$b = \var{binop}\bcode{-float}\ r_d\ r_{s_1}\ r_{s_2}$}
  \AxiomC{$\registers(r_{s_1}) \subtype \datafloat$}
  \AxiomC{$\registers(r_{s_2}) \subtype \datafloat$}
  \TrinaryInfC{$Q' = (\registers[r_d \mapsto \datafloat], \rvoid)$}
\end{sequent}

\begin{sequent}
  \AxiomC{$b = \var{binop}\bcode{-float/2addr}\ r_d\ r_s$}
  \AxiomC{$\registers(r_d) \subtype \datafloat$}
  \AxiomC{$\registers(r_s) \subtype \datafloat$}
  \TrinaryInfC{$Q' = (\registers[r_d \mapsto \datafloat], \rvoid)$}
\end{sequent}

\begin{sequent}
  \AxiomC{$b = \bcode{\{shl,shr,ushr\}}\bcode{-long}\ r_d\ r_{s_1}\ r_{s_2}$}
  \AxiomC{$\registers(r_{s_1}) \subtype \datalong$ \defaultHypSeparation $\registers(r_{s_1}) = \registers(r_{s_1}+1)$}
  \AxiomC{$\registers(r_{s_2}) \subtype \dataint$ \defaultHypSeparation $\registers(r_{s_2}) = \registers(r_{s_2}+1)$}
  \TrinaryInfC{$Q' = (\registers[r_d \mapsto \datalong, r_d + 1\mapsto \datalong], \rvoid)$}
\end{sequent}

\begin{sequent}
  \AxiomC{$b = \var{binopLogic}\bcode{-long}\ r_d\ r_{s_1}\ r_{s_2}$}
  \AxiomC{$\registers(r_{s_1}) \subtype \datalong$ \defaultHypSeparation $\registers(r_{s_1}) = \registers(r_{s_1}+1)$}
  \AxiomC{$\registers(r_{s_2}) \subtype \datalong$ \defaultHypSeparation $\registers(r_{s_2}) = \registers(r_{s_2}+1)$}
  \TrinaryInfC{$Q' = (\registers[r_d \mapsto \registers(r_{s_1}) \vee \registers(r_{s_2}), r_d + 1\mapsto \registers(r_{s_1}+1) \vee \registers(r_{s_2}+1)], \rvoid)$}
\end{sequent}

\begin{sequent}
  \AxiomC{$b = \var{binopArith}\bcode{-long}\ r_d\ r_{s_1}\ r_{s_2}$}
  \AxiomC{$\registers(r_{s_1}) \subtype \datalong$ \defaultHypSeparation $\registers(r_{s_1}) = \registers(r_{s_1}+1)$}
  \AxiomC{$\registers(r_{s_2}) \subtype \datalong$ \defaultHypSeparation $\registers(r_{s_2}) = \registers(r_{s_2}+1)$}
  \TrinaryInfC{$Q' = (\registers[r_d \mapsto \datalong, r_d + 1\mapsto \datalong], \rvoid)$}
\end{sequent}

\begin{sequent}
  \AxiomC{$b = \bcode{\{shl,shr,ushr\}}\bcode{-long/2addr}\ r_d\ r_s$}
  \AxiomC{$\registers(r_d) \subtype \datalong$ \defaultHypSeparation $\registers(r_d) = \registers(r_d+1)$}
  \AxiomC{$\registers(r_s) \subtype \dataint$ \defaultHypSeparation $\registers(r_s) = \registers(r_s+1)$}
  \TrinaryInfC{$Q' = (\registers[r_d \mapsto \datalong, r_d + 1\mapsto \datalong], \rvoid)$}
\end{sequent}

\begin{sequent}
  \AxiomC{$b = \var{binopLogic}\bcode{-long/2addr}\ r_d\ r_s$}
  \AxiomC{$\registers(r_d) \subtype \datalong$ \defaultHypSeparation $\registers(r_d) = \registers(r_d + 1)$}
  \AxiomC{$\registers(r_s) \subtype \datalong$ \defaultHypSeparation $\registers(r_s) = \registers(r_s + 1)$}
  \TrinaryInfC{$Q' = (\registers[r_d \mapsto \registers(r_d) \vee \registers(r_s), r_d + 1\mapsto \registers(r_d+1) \vee \registers(r_s+1)], \rvoid)$}
\end{sequent}

\begin{sequent}
  \AxiomC{$b = \var{binopArith}\bcode{-long/2addr}\ r_d\ r_s$}
  \AxiomC{$\registers(r_d) \subtype \datalong$ \defaultHypSeparation $\registers(r_d) = \registers(r_d + 1)$}
  \AxiomC{$\registers(r_s) \subtype \datalong$ \defaultHypSeparation $\registers(r_s) = \registers(r_s + 1)$}
  \TrinaryInfC{$Q' = (\registers[r_d \mapsto \datalong, r_d + 1\mapsto \datalong], \rvoid)$}
\end{sequent}

\begin{sequent}
  \AxiomC{$b = \var{binopArith}\bcode{-int}\ r_d\ r_{s_1}\ r_{s_2}$}
  \AxiomC{$\registers(r_{s_1}) \subtype \dataint$}
  \AxiomC{$\registers(r_{s_2}) \subtype \dataint$}
  \TrinaryInfC{$Q' = (\registers[r_d \mapsto \dataint], \rvoid)$}
\end{sequent}

\begin{sequent}
  \AxiomC{$b = \var{binopArith}\bcode{-int/2addr}\ r_d\ r_s$}
  \AxiomC{$\registers(r_d) \subtype \dataint$}
  \AxiomC{$\registers(r_s) \subtype \dataint$}
  \TrinaryInfC{$Q' = (\registers[r_d \mapsto \dataint], \rvoid)$}
\end{sequent}

\begin{sequent}
  \AxiomC{$b = \var{binopLogic}\bcode{-int}\ r_d\ r_{s_1}\ r_{s_2}$}
  \AxiomC{$\registers(r_{s_1}) \subtype \dataint$}
  \AxiomC{$\registers(r_{s_2}) \subtype \dataint$}
  \TrinaryInfC{$Q' = (\registers[r_d \mapsto \registers(r_{s_1}) \vee \registers(r_{s_2})], \rvoid)$}
\end{sequent}

\begin{sequent}
  \AxiomC{$b = \var{binopLogic}\bcode{-int/2addr}\ r_d\ r_s$}
  \AxiomC{$\registers(r_d) \subtype \dataint$}
  \AxiomC{$\registers(r_s) \subtype \dataint$}
  \TrinaryInfC{$Q' = (\registers[r_d \mapsto \registers(r_d) \vee \registers(r_s)], \rvoid)$}
\end{sequent}

\begin{sequent}
  \AxiomC{$b = \var{binop}\bcode{-int/lit16}\ r_d\ r_s\ n$}
  \AxiomC{$\registers(r_s) \subtype \dataint$}
  \BinaryInfC{$Q' = (\registers[r_d \mapsto \dataint], \rvoid)$}
\end{sequent}

\begin{sequent}
  \AxiomC{$b = \var{binop}\bcode{-int/lit8}\ r_d\ r_s\ n$}
  \AxiomC{$\registers(r_s) \subtype \dataint$}
  \BinaryInfC{$Q' = (\registers[r_d \mapsto \dataint], \rvoid)$}
\end{sequent}


\begin{sequent}
  \AxiomC{$b = \bcode{check-cast}\ r_s\ t$}
  \AxiomC{$\registers(r_s) \subtype \refobject{\jobject}$}
  \AxiomC{$t \subtype \refobject{\jobject}$}
  \TrinaryInfC{$Q' = (\registers, \rvoid)$}\todo{le cas où le check cast (ne) passe (pas) doit être dans le flot de contrôle}
\end{sequent}

\begin{sequent}
  \AxiomC{$b = \bcode{cmp-double}\ r_{s_1}\ r_{s_2}$}
  \AxiomC{$\registers(r_{s_1}) \subtype \datadouble$ \defaultHypSeparation $\registers(r_{s_1}) = \registers(r_{s_1} + 1)$}
  \AxiomC{$\registers(r_{s_2}) \subtype \datadouble$ \defaultHypSeparation $\registers(r_{s_2}) = \registers(r_{s_2} + 1)$}
  \TrinaryInfC{$Q' = (\registers[r_d \mapsto \databyte], \rvoid)$}
\end{sequent}

\begin{sequent}
  \AxiomC{$b = \bcode{cmp-float}\ r_{s_1}\ r_{s_2}$}
  \AxiomC{$\registers(r_{s_1}) \subtype \datafloat$}
  \AxiomC{$\registers(r_{s_2}) \subtype \datafloat$}
  \TrinaryInfC{$Q' = (\registers[r_d \mapsto \databyte], \rvoid)$}
\end{sequent}

\begin{sequent}
  \AxiomC{$b = \bcode{cmp-long}\ r_{s_1}\ r_{s_2}$}
  \AxiomC{$\registers(r_{s_1}) \subtype \datalong$ \defaultHypSeparation $\registers(r_{s_1}) = \registers(r_{s_1} + 1)$}
  \AxiomC{$\registers(r_{s_2}) \subtype \datalong$ \defaultHypSeparation $\registers(r_{s_2}) = \registers(r_{s_2} + 1)$}
  \TrinaryInfC{$Q' = (\registers[r_d \mapsto \databyte], \rvoid)$}
\end{sequent}

\begin{sequent}
  \AxiomC{$b = \bcode{const}\ r_d\ n$}
  \AxiomC{$n \neq 0$}
  \BinaryInfC{$Q' = (\registers[r_d \mapsto \meettd], \rvoid)$}
\end{sequent}

\begin{sequent}
  \AxiomC{$b = \bcode{const}\ r_d\ 0$}
  \UnaryInfC{$Q' = (\registers[r_d \mapsto \meetz], \rvoid)$}
\end{sequent}

\begin{sequent}
  \AxiomC{$b = \bcode{const-class}\ r_d\ t$}
  \AxiomC{$t \subtype \refobject{\jobject}$}
  \BinaryInfC{$Q' = (\registers[r_d \mapsto \refobject{\jclass}], \rvoid)$}
\end{sequent}

\begin{sequent}
  \AxiomC{$b = \bcode{const-string}\ r_d\ s$}
  \UnaryInfC{$Q' = (\registers[r_d \mapsto \refobject{\jstring}], \rvoid)$}
\end{sequent}

\begin{sequent}
  \AxiomC{$b = \bcode{const-wide}\ r_d\ n$}
  \UnaryInfC{$Q' = (\registers[r_d \mapsto \meetsq, r_d + 1 \mapsto \meetsq], \rvoid)$}
\end{sequent}

\begin{sequent}
  \AxiomC{$b = \bcode{fill-array-data}\ r_a\ n$}\todo{vérifier la sémantique}
  \AxiomC{$\registers(r_a) \subtype \pnull$}
  \BinaryInfC{$Q' = (\registers, \rvoid)$}
\end{sequent}

\begin{sequent}
  \AxiomC{$b = \bcode{fill-array-data}\ r_a\ n$}\todo{vérifier la sémantique}
  \AxiomC{$\registers(r_a) = \refarray{1}{t}$}
  \BinaryInfC{$Q' = (\registers, \rvoid)$}
\end{sequent}

\begin{sequent}
  \AxiomC{$b = \bcode{filled-new-array}\ r_s\ t$}\todo{vérifier la sémantique}
  \AxiomC{$t \subtype \pnull$}
  \AxiomC{$\registers(r_s) \subtype \dataint$}
  \TrinaryInfC{$Q' = (\registers, t)$}
\end{sequent}

\begin{sequent}
  \AxiomC{$b = \bcode{filled-new-array}\ r_s\ t$}\todo{vérifier la sémantique}
  \AxiomC{$t = \refarray{1}{t'}$ \defaultHypSeparation $t' \subtype \joinz$}
  \AxiomC{$\registers(r_s) \subtype \dataint$}
  \TrinaryInfC{$Q' = (\registers, t)$}
\end{sequent}

\begin{sequent}
  \AxiomC{$b = \bcode{filled-new-array-range}\ ?????????????????\ t$}\todo{todo}
  \UnaryInfC{$Q' = (\registers, t)$}
\end{sequent}


\begin{sequent}
  \AxiomC{$b = \bcode{double-to-int}\ r_d\ r_s$}
  \AxiomC{$\registers(r_s) \subtype \datadouble$}
  \AxiomC{$\registers(r_s) = \registers(r_s + 1)$}
  \TrinaryInfC{$Q' = (\registers[r_d \mapsto \dataint], \rvoid)$}
\end{sequent}

\begin{sequent}
  \AxiomC{$b = \bcode{double-to-float}\ r_d\ r_s$}
  \AxiomC{$\registers(r_s) \subtype \datadouble$}
  \AxiomC{$\registers(r_s) = \registers(r_s + 1)$}
  \TrinaryInfC{$Q' = (\registers[r_d \mapsto \datafloat], \rvoid)$}
\end{sequent}

\begin{sequent}
  \AxiomC{$b = \bcode{double-to-long}\ r_d\ r_s$}
  \AxiomC{$\registers(r_s) \subtype \datadouble$}
  \AxiomC{$\registers(r_s) = \registers(r_s + 1)$}
  \TrinaryInfC{$Q' = (\registers[r_d \mapsto \datalong, r_d + 1\mapsto \datalong], \rvoid)$}
\end{sequent}


\begin{sequent}
  \AxiomC{$b = \bcode{float-to-double}\ r_d\ r_s$}
  \AxiomC{$\registers(r_s) \subtype \datafloat$}
  \BinaryInfC{$Q' = (\registers[r_d \mapsto \datadouble, r_d + 1\mapsto \datadouble], \rvoid)$}
\end{sequent}

\begin{sequent}
  \AxiomC{$b = \bcode{float-to-int}\ r_d\ r_s$}
  \AxiomC{$\registers(r_s) \subtype \datafloat$}
  \BinaryInfC{$Q' = (\registers[r_d \mapsto \dataint], \rvoid)$}
\end{sequent}

\begin{sequent}
  \AxiomC{$b = \bcode{float-to-long}\ r_d\ r_s$}
  \AxiomC{$\registers(r_s) \subtype \datafloat$}
  \BinaryInfC{$Q' = (\registers[r_d \mapsto \datalong, r_d + 1\mapsto \datalong], \rvoid)$}
\end{sequent}



\begin{sequent}
  \AxiomC{$b = \bcode{goto}\ n$}
  \UnaryInfC{$Q' = (\registers, \rvoid)$}
\end{sequent}



\begin{sequent}
  \AxiomC{$b = \bcode{if-eq}\ r_{s_1}\ r_{s_2}\ n$}
  \AxiomC{$(\registers(r_{s_1}) \vee \registers(r_{s_2})) \subtype \{ \dataint, \datafloat, \refobject{\jobject} \}$}
  \BinaryInfC{$Q' = (\registers, \rvoid)$}
\end{sequent}

\begin{sequent}
  \AxiomC{$b = \bcode{if-ne}\ r_{s_1}\ r_{s_2}\ n$}
  \AxiomC{$(\registers(r_{s_1}) \vee \registers(r_{s_2})) \subtype \{ \dataint, \datafloat, \refobject{\jobject} \}$}
  \BinaryInfC{$Q' = (\registers, \rvoid)$}
\end{sequent}

\begin{sequent}
  \AxiomC{$b = \bcode{if-lt}\ r_{s_1}\ r_{s_2}\ n$}
  \AxiomC{$(\registers(r_{s_1}) \vee \registers(r_{s_2})) \subtype \{ \dataint, \datafloat \}$}
  \BinaryInfC{$Q' = (\registers, \rvoid)$}
\end{sequent}

\begin{sequent}
  \AxiomC{$b = \bcode{if-ge}\ r_{s_1}\ r_{s_2}\ n$}
  \AxiomC{$(\registers(r_{s_1}) \vee \registers(r_{s_2})) \subtype \{ \dataint, \datafloat \}$}
  \BinaryInfC{$Q' = (\registers, \rvoid)$}
\end{sequent}

\begin{sequent}
  \AxiomC{$b = \bcode{if-gt}\ r_{s_1}\ r_{s_2}\ n$}
  \AxiomC{$(\registers(r_{s_1}) \vee \registers(r_{s_2})) \subtype \{ \dataint, \datafloat \}$}
  \BinaryInfC{$Q' = (\registers, \rvoid)$}
\end{sequent}

\begin{sequent}
  \AxiomC{$b = \bcode{if-le}\ r_{s_1}\ r_{s_2}\ n$}
  \AxiomC{$(\registers(r_{s_1}) \vee \registers(r_{s_2})) \subtype \{ \dataint, \datafloat \}$}
  \BinaryInfC{$Q' = (\registers, \rvoid)$}
\end{sequent}

\begin{sequent}
  \AxiomC{$b = \bcode{if-eqz}\ r_s\ n$}
  \AxiomC{$\registers(r_s) \subtype \{ \dataint, \datafloat, \refobject{\jobject} \}$}
  \BinaryInfC{$Q' = (\registers, \rvoid)$}
\end{sequent}

\begin{sequent}
  \AxiomC{$b = \bcode{if-nez}\ r_s\ n$}
  \AxiomC{$\registers(r_s) \subtype \{ \dataint, \datafloat, \refobject{\jobject} \}$}
  \BinaryInfC{$Q' = (\registers, \rvoid)$}
\end{sequent}

\begin{sequent}
  \AxiomC{$b = \bcode{if-ltz}\ r_s\ n$}
  \AxiomC{$\registers(r_s) \subtype \{ \dataint, \datafloat \}$}
  \BinaryInfC{$Q' = (\registers, \rvoid)$}
\end{sequent}

\begin{sequent}
  \AxiomC{$b = \bcode{if-gez}\ r_s\ n$}
  \AxiomC{$\registers(r_s) \subtype \{ \dataint, \datafloat \}$}
  \BinaryInfC{$Q' = (\registers, \rvoid)$}
\end{sequent}

\begin{sequent}
  \AxiomC{$b = \bcode{if-gtz}\ r_s\ n$}
  \AxiomC{$\registers(r_s) \subtype \{ \dataint, \datafloat \}$}
  \BinaryInfC{$Q' = (\registers, \rvoid)$}
\end{sequent}

\begin{sequent}
  \AxiomC{$b = \bcode{if-lez}\ r_s\ n$}
  \AxiomC{$\registers(r_s) \subtype \{ \dataint, \datafloat \}$}
  \BinaryInfC{$Q' = (\registers, \rvoid)$}
\end{sequent}


\begin{sequent}
  \AxiomC{$b = \bcode{iget}\ r_d\ r_o\ c'.f$}
  \AxiomC{$\typeof(c'.f) \in \{ \dataint, \datafloat \}$}
  \AxiomC{$\registers(r_o) \subtype \refobject{c'}$ \defaultHypSeparation $\checkaccess(c.m, b)$}
  \TrinaryInfC{$Q' = (\registers[r_d \mapsto \typeof(c'.f)], \rvoid)$}
\end{sequent}

\begin{sequent}
  \AxiomC{$b = \bcode{iget-boolean}\ r_d\ r_o\ c'.f$}
  \AxiomC{$\typeof(c'.f) = \databool$}
  \AxiomC{$\registers(r_o) \subtype \refobject{c'}$ \defaultHypSeparation $\checkaccess(c.m, b)$}
  \TrinaryInfC{$Q' = (\registers[r_d \mapsto \typeof(c'.f)], \rvoid)$}
\end{sequent}

\begin{sequent}
  \AxiomC{$b = \bcode{iget-byte}\ r_d\ r_o\ c'.f$}
  \AxiomC{$\typeof(c'.f) = \databyte$}
  \AxiomC{$\registers(r_o) \subtype \refobject{c'}$ \defaultHypSeparation $\checkaccess(c.m, b)$}
  \TrinaryInfC{$Q' = (\registers[r_d \mapsto \typeof(c'.f)], \rvoid)$}
\end{sequent}

\begin{sequent}
  \AxiomC{$b = \bcode{iget-char}\ r_d\ r_o\ c'.f$}
  \AxiomC{$\typeof(c'.f) = \datachar$}
  \AxiomC{$\registers(r_o) \subtype \refobject{c'}$ \defaultHypSeparation $\checkaccess(c.m, b)$}
  \TrinaryInfC{$Q' = (\registers[r_d \mapsto \typeof(c'.f)], \rvoid)$}
\end{sequent}

\begin{sequent}
  \AxiomC{$b = \bcode{iget-object}\ r_d\ r_o\ c'.f$}
  \AxiomC{$\typeof(c'.f) \subtype \refobject{\jobject}$}
  \AxiomC{$\registers(r_o) \subtype \refobject{c'}$ \defaultHypSeparation $\checkaccess(c.m, b)$}
  \TrinaryInfC{$Q' = (\registers[r_d \mapsto \typeof(c'.f)], \rvoid)$}
\end{sequent}

\begin{sequent}
  \AxiomC{$b = \bcode{iget-short}\ r_d\ r_o\ c'.f$}
  \AxiomC{$\typeof(c'.f) = \datashort$}
  \AxiomC{$\registers(r_o) \subtype \refobject{c'}$ \defaultHypSeparation $\checkaccess(c.m, b)$}
  \TrinaryInfC{$Q' = (\registers[r_d \mapsto \typeof(c'.f)], \rvoid)$}
\end{sequent}

\begin{sequent}
  \AxiomC{$b = \bcode{iget-wide}\ r_d\ r_o\ c'.f$}
  \AxiomC{$\typeof(c'.f) \subtype \joinsq$}
  \AxiomC{$\registers(r_o) \subtype \refobject{c'}$ \defaultHypSeparation $\checkaccess(c.m, b)$}
  \TrinaryInfC{$Q' = (\registers[r_d \mapsto \typeof(c'.f), r_d + 1 \mapsto \typeof(c'.f)], \rvoid)$}
\end{sequent}



\begin{sequent}
  \AxiomC{$b = \bcode{instance-of}\ r_d\ r_s\ t$}
  \AxiomC{$t \subtype \refobject{\jobject}$}
  \AxiomC{$\registers(r_s) \subtype \refobject{\jobject}$}
  \TrinaryInfC{$Q' = (\registers[r_d \mapsto \databool], \rvoid)$}
\end{sequent}

\begin{sequent}
  \AxiomC{$b = \bcode{int-to-byte}\ r_d\ r_s$}
  \AxiomC{$\registers(r_s) \subtype \dataint$}
  \BinaryInfC{$Q' = (\registers[r_d \mapsto \databyte], \rvoid)$}
\end{sequent}

\begin{sequent}
  \AxiomC{$b = \bcode{int-to-char}\ r_d\ r_s$}
  \AxiomC{$\registers(r_s) \subtype \dataint$}
  \BinaryInfC{$Q' = (\registers[r_d \mapsto \datachar], \rvoid)$}
\end{sequent}

\begin{sequent}
  \AxiomC{$b = \bcode{int-to-double}\ r_d\ r_s$}
  \AxiomC{$\registers(r_s) \subtype \dataint$}
  \BinaryInfC{$Q' = (\registers[r_d \mapsto \datadouble, r_d + 1\mapsto \datadouble], \rvoid)$}
\end{sequent}

\begin{sequent}
  \AxiomC{$b = \bcode{int-to-float}\ r_d\ r_s$}
  \AxiomC{$\registers(r_s) \subtype \dataint$}
  \BinaryInfC{$Q' = (\registers[r_d \mapsto \datafloat], \rvoid)$}
\end{sequent}

\begin{sequent}
  \AxiomC{$b = \bcode{int-to-long}\ r_d\ r_s$}
  \AxiomC{$\registers(r_s) \subtype \dataint$}
  \BinaryInfC{$Q' = (\registers[r_d \mapsto \datalong, r_d + 1\mapsto \datalong], \rvoid)$}
\end{sequent}

\begin{sequent}
  \AxiomC{$b = \bcode{int-to-short}\ r_d\ r_s$}
  \AxiomC{$\registers(r_s) \subtype \dataint$}
  \BinaryInfC{$Q' = (\registers[r_d \mapsto \datashort], \rvoid)$}
\end{sequent}

\begin{sequent}
  \AxiomC{$b = \bcode{invoke-direct}\ r_i\ r_1 \dots r_n\ c'.m'$}\todo{checker si private ou si constructeur !!!}
  \AxiomC{$\registers(r_i) \subtype \refobject{c'}$ \defaultHypSeparation $\checkaccess(c.m, b)$}
  \AxiomC{$\nbparams(c'.m') = n$ \defaultHypSeparation $\forall 1 \leq j \leq n\ \registers(r_j) \subtype \parameter(c'.m', j)$}
  \TrinaryInfC{$Q' = (\registers, \returnof(c'.m'))$}
\end{sequent}

\begin{sequent}
  \AxiomC{$b = \bcode{invoke-interface}\ r_i\ r_1 \dots r_n\ c'.m'$}\todo{interêt de checker si c'est bien une interface ???}
  \AxiomC{$\registers(r_i) \subtype \refobject{c'}$ \defaultHypSeparation $\checkaccess(c.m, b)$}
  \AxiomC{$\nbparams(c'.m') = n$ \defaultHypSeparation $\forall 1 \leq j \leq n\ \registers(r_j) \subtype \parameter(c'.m', j)$}
  \TrinaryInfC{$Q' = (\registers, \returnof(c'.m'))$}
\end{sequent}

\begin{sequent}
  \AxiomC{$b = \bcode{invoke-static}\ r_1 \dots r_n\ c'.m'$}
  \AxiomC{$\checkaccess(c.m, b)$}
  \AxiomC{$\nbparams(c'.m') = n$ \defaultHypSeparation $\forall 1 \leq j \leq n\ \registers(r_j) \subtype \parameter(c'.m', j)$}
  \TrinaryInfC{$Q' = (\registers, \returnof(c'.m'))$}
\end{sequent}

\begin{sequent}
  \AxiomC{$b = \bcode{invoke-super}\ r_i\ r_1 \dots r_n\ c'.m'$}
  \AxiomC{$\registers(r_i) \subtype \refobject{c'}$ \defaultHypSeparation $\checkaccess(c.m, b)$}
  \AxiomC{$\nbparams(c'.m') = n$ \defaultHypSeparation $\forall 1 \leq j \leq n\ \registers(r_j) \subtype \parameter(c'.m', j)$}
  \TrinaryInfC{$Q' = (\registers, \returnof(c'.m'))$}
\end{sequent}

\begin{sequent}
  \AxiomC{$b = \bcode{invoke-virtual}\ r_i\ r_1 \dots r_n\ c'.m'$}
  \AxiomC{$\registers(r_i) \subtype \refobject{c'}$ \defaultHypSeparation $\checkaccess(c.m, b)$}
  \AxiomC{$\nbparams(c'.m') = n$ \defaultHypSeparation $\forall 1 \leq j \leq n\ \registers(r_j) \subtype \parameter(c'.m', j)$}
  \TrinaryInfC{$Q' = (\registers, \returnof(c'.m'))$}
\end{sequent}






\begin{sequent}
  \AxiomC{$b = \bcode{iput}\ r_s\ r_o\ c'.f$}\todo{ACCESSIBLE ? FINAL ??? et pour les iget et les suivnats aussi !!!!!}
  \AxiomC{$\registers(r_s) \subtype \typeof(c'.f) \subtype \jointd$}
  \AxiomC{$\registers(r_o) \subtype \refobject{c'}$ \defaultHypSeparation $\checkaccess(c.m, b)$}
  \TrinaryInfC{$Q' = (\registers, \rvoid)$}
\end{sequent}

\begin{sequent}
  \AxiomC{$b = \bcode{iput-boolean}\ r_s\ r_o\ c'.f$}
  \AxiomC{$\typeof(c'.f) = \databool$ \defaultHypSeparation $\registers(r_s) \subtype \typeof(c'.f)$}
  \AxiomC{$\registers(r_o) \subtype \refobject{c'}$ \defaultHypSeparation $\checkaccess(c.m, b)$}
  \TrinaryInfC{$Q' = (\registers, \rvoid)$}
\end{sequent}

\begin{sequent}
  \AxiomC{$b = \bcode{iput-byte}\ r_s\ r_o\ c'.f$}
  \AxiomC{$\typeof(c'.f) = \databyte$ \defaultHypSeparation $\registers(r_s) \subtype \typeof(c'.f)$}
  \AxiomC{$\registers(r_o) \subtype \refobject{c'}$ \defaultHypSeparation $\checkaccess(c.m, b)$}
  \TrinaryInfC{$Q' = (\registers, \rvoid)$}
\end{sequent}

\begin{sequent}
  \AxiomC{$b = \bcode{iput-char}\ r_s\ r_o\ c'.f$}
  \AxiomC{$\typeof(c'.f) = \datachar$ \defaultHypSeparation $\registers(r_s) \subtype \typeof(c'.f)$}
  \AxiomC{$\registers(r_o) \subtype \refobject{c'}$ \defaultHypSeparation $\checkaccess(c.m, b)$}
  \TrinaryInfC{$Q' = (\registers, \rvoid)$}
\end{sequent}

\begin{sequent}
  \AxiomC{$b = \bcode{iput-object}\ r_s\ r_o\ c'.f$}
  \AxiomC{$\typeof(c'.f) \subtype \refobject{\jobject}$ \defaultHypSeparation $\registers(r_s) \subtype \typeof(c'.f)$}
  \AxiomC{$\registers(r_o) \subtype \refobject{c'}$ \defaultHypSeparation $\checkaccess(c.m, b)$}
  \TrinaryInfC{$Q' = (\registers, \rvoid)$}
\end{sequent}

\begin{sequent}
  \AxiomC{$b = \bcode{iput-short}\ r_s\ r_o\ c'.f$}
  \AxiomC{$\typeof(c'.f) = \datashort$ \defaultHypSeparation $\registers(r_s) \subtype \typeof(c'.f)$}
  \AxiomC{$\registers(r_o) \subtype \refobject{c'}$ \defaultHypSeparation $\checkaccess(c.m, b)$}
  \TrinaryInfC{$Q' = (\registers, \rvoid)$}
\end{sequent}

\begin{sequent}
  \AxiomC{$b = \bcode{iput-wide}\ r_s\ r_o\ c'.f$}
  \AxiomC{$\registers(r_s) \subtype \typeof(c'.f) \subtype \joinsq$ \defaultHypSeparation $\registers(r_s) = \registers(r_s + 1)$}
  \AxiomC{$\registers(r_o) \subtype \refobject{c'}$ \defaultHypSeparation $\checkaccess(c.m, b)$}
  \TrinaryInfC{$Q' = (\registers, \rvoid)$}
\end{sequent}



\begin{sequent}
  \AxiomC{$b = \bcode{long-to-int}\ r_d\ r_s$}
  \AxiomC{$\registers(r_s) \subtype \datalong$}
  \AxiomC{$\registers(r_s) = \registers(r_s + 1)$}
  \TrinaryInfC{$Q' = (\registers[r_d \mapsto \dataint], \rvoid)$}
\end{sequent}

\begin{sequent}
  \AxiomC{$b = \bcode{long-to-float}\ r_d\ r_s$}
  \AxiomC{$\registers(r_s) \subtype \datalong$}
  \AxiomC{$\registers(r_s) = \registers(r_s + 1)$}
  \TrinaryInfC{$Q' = (\registers[r_d \mapsto \datafloat], \rvoid)$}
\end{sequent}

\begin{sequent}
  \AxiomC{$b = \bcode{long-to-double}\ r_d\ r_s$}
  \AxiomC{$\registers(r_s) \subtype \datalong$}
  \AxiomC{$\registers(r_s) = \registers(r_s + 1)$}
  \TrinaryInfC{$Q' = (\registers[r_d \mapsto \datadouble, r_d + 1\mapsto \datadouble], \rvoid)$}
\end{sequent}



\begin{sequent}
  \AxiomC{$b = \bcode{monitor-*}\ r_s$}
  \AxiomC{$\registers(r_s) \subtype \refobject{\jobject}$}
  \BinaryInfC{$Q' = (\registers, \rvoid)$}
\end{sequent}

\begin{sequent}
  \AxiomC{$b = \bcode{move}\ r_d\ r_s$}
  \AxiomC{$\registers(r_s) \subtype \jointd$}
  \BinaryInfC{$Q' = (\registers[r_d \mapsto \registers(r_s)], \rvoid)$}
\end{sequent}

\begin{sequent}
  \AxiomC{$b = \bcode{move-object}\ r_d\ r_s$}
  \AxiomC{$\registers(r_s) \subtype \refobject{\jobject}$}
  \BinaryInfC{$Q' = (\registers[r_d \mapsto \registers(r_s)], \rvoid)$}
\end{sequent}

\begin{sequent}
  \AxiomC{$b = \bcode{move-wide}\ r_d\ r_s$}
  \AxiomC{$\registers(r_s) \subtype \joinsq$ \defaultHypSeparation $\registers(r_s) = \registers(r_s+1)$}
  \BinaryInfC{$Q' = (\registers[r_d \mapsto \registers(r_s), r_d + 1 \mapsto \registers(r_s + 1)], \rvoid)$}
\end{sequent}

\begin{sequent}
  \AxiomC{$b = \bcode{move-result}\ r_d$}
  \AxiomC{$\lastresult \neq \rvoid$ \defaultHypSeparation $\lastresult \subtype \jointd$}
  \BinaryInfC{$Q' = (\registers[r_d \mapsto \lastresult], \rvoid)$}
\end{sequent}

\begin{sequent}
  \AxiomC{$b = \bcode{move-result-object}\ r_d$}
  \AxiomC{$\lastresult \neq \rvoid$ \defaultHypSeparation $\lastresult \subtype \refobject{\jobject}$}
  \BinaryInfC{$Q' = (\registers[r_d \mapsto \lastresult], \rvoid)$}
\end{sequent}

\begin{sequent}
  \AxiomC{$b = \bcode{move-result-wide}\ r_d$}
  \AxiomC{$\lastresult \neq \rvoid$ \defaultHypSeparation $\lastresult \subtype \joinsq$}
  \BinaryInfC{$Q' = (\registers[r_d \mapsto \lastresult, r_d + 1 \mapsto \lastresult], \rvoid)$}
\end{sequent}

\begin{sequent}
  \AxiomC{$b = \bcode{move-exception}\ r_d$}
  \AxiomC{$\lastresult \neq \rvoid$ \defaultHypSeparation $\lastresult \subtype \refobject{\jexception}$}
  \BinaryInfC{$Q' = (\registers[r_d \mapsto \lastresult], \rvoid)$}
\end{sequent}


\begin{sequent}
  \AxiomC{$b = \bcode{neg-double}\ r_d\ r_s$}
  \AxiomC{$\registers(r_s) \subtype \datadouble$}
  \AxiomC{$\registers(r_s) = \registers(r_s + 1)$}
  \TrinaryInfC{$Q' = (\registers[r_d \mapsto \datadouble, r_d + 1\mapsto \datadouble], \rvoid)$}
\end{sequent}

\begin{sequent}
  \AxiomC{$b = \bcode{neg-float}\ r_d\ r_s$}
  \AxiomC{$\registers(r_s) \subtype \datafloat$}
  \BinaryInfC{$Q' = (\registers[r_d \mapsto \datafloat], \rvoid)$}
\end{sequent}

\begin{sequent}
  \AxiomC{$b = \bcode{neg-int}\ r_d\ r_s$}
  \AxiomC{$\registers(r_s) \subtype \dataint$}
  \BinaryInfC{$Q' = (\registers[r_d \mapsto \dataint], \rvoid)$}
\end{sequent}

\begin{sequent}
  \AxiomC{$b = \bcode{neg-long}\ r_d\ r_s$}
  \AxiomC{$\registers(r_s) \subtype \datalong$}
  \AxiomC{$\registers(r_s) = \registers(r_s + 1)$}
  \TrinaryInfC{$Q' = (\registers[r_d \mapsto \datalong, r_d + 1\mapsto \datalong], \rvoid)$}
\end{sequent}

\begin{sequent}
  \AxiomC{$b = \bcode{new-array}\ r_d\ r_s\ t$}
  \AxiomC{$t = \refarray{p}{t'}$}
  \AxiomC{$\registers(r_s) \subtype \dataint$}
  \TrinaryInfC{$Q' = (\registers[r_d \mapsto t], \rvoid)$}
\end{sequent}

\begin{sequent}
  \AxiomC{$b = \bcode{new-instance}\ r_d\ t$}
  \AxiomC{$t = \refobject{c'}$}
  \BinaryInfC{$Q' = (\registers[r_d \mapsto t], \rvoid)$}
\end{sequent}


\begin{sequent}
  \AxiomC{$b = \bcode{nop}$}
  \UnaryInfC{$Q' = (\registers, \rvoid)$}
\end{sequent}


\begin{sequent}
  \AxiomC{$b = \bcode{not-int}\ r_d\ r_s$}
  \AxiomC{$\registers(r_s) \subtype \dataint$}
  \BinaryInfC{$Q' = (\registers[r_d \mapsto \dataint], \rvoid)$}
\end{sequent}

\begin{sequent}
  \AxiomC{$b = \bcode{not-long}\ r_d\ r_s$}
  \AxiomC{$\registers(r_s) \subtype \datalong$}
  \AxiomC{$\registers(r_s) = \registers(r_s + 1)$}
  \TrinaryInfC{$Q' = (\registers[r_d \mapsto \datalong, r_d + 1\mapsto \datalong], \rvoid)$}
\end{sequent}

\begin{sequent}
  \AxiomC{$b = \bcode{packed-switch}\ r_s\ n$}
  \AxiomC{$\registers(r_s) \subtype \dataint$}
  \BinaryInfC{$Q' = (\registers, \rvoid)$}
\end{sequent}

\begin{sequent}
  \AxiomC{$b = \bcode{return-void}$}
  \AxiomC{$\returnof(m) = \rvoid$}
  \BinaryInfC{$Q' = (\registers, \rvoid)$}
\end{sequent}

\begin{sequent}
  \AxiomC{$b = \bcode{return}\ r_s$}
  \AxiomC{$\registers(r_s) \subtype \returnof(m) \subtype \jointd$}
  \BinaryInfC{$Q' = (\registers, \rvoid)$}
\end{sequent}

\begin{sequent}
  \AxiomC{$b = \bcode{return-object}\ r_s$}
  \AxiomC{$\registers(r_s) \subtype \returnof(m) \subtype \refobject{\jobject}$}
  \BinaryInfC{$Q' = (\registers, \rvoid)$}
\end{sequent}

\begin{sequent}
  \AxiomC{$b = \bcode{return-wide}\ r_s$}
  \AxiomC{$\registers(r_s) \subtype \returnof(m) \subtype \joinsq$}
  \AxiomC{$\registers(r_s) = \registers(r_s + 1)$}
  \TrinaryInfC{$Q' = (\registers, \rvoid)$}
\end{sequent}



\begin{sequent}
  \AxiomC{$b = \bcode{sparse-switch}\ r_s\ n$}
  \AxiomC{$\registers(r_s) \subtype \dataint$}
  \BinaryInfC{$Q' = (\registers, \rvoid)$}
\end{sequent}

\begin{sequent}
  \AxiomC{$b = \bcode{sget}\ r_d\ c'.f$}
  \AxiomC{$\typeof(c'.f) \in \{ \dataint, \datafloat \}$}
  \AxiomC{$\checkaccess(c.m, b)$}
  \TrinaryInfC{$Q' = (\registers[r_d \mapsto \typeof(c'.f)], \rvoid)$}
\end{sequent}

\begin{sequent}
  \AxiomC{$b = \bcode{sget-boolean}\ r_d\ c'.f$}
  \AxiomC{$\typeof(c'.f) = \databool$}
  \AxiomC{$\checkaccess(c.m, b)$}
  \TrinaryInfC{$Q' = (\registers[r_d \mapsto \typeof(c'.f)], \rvoid)$}
\end{sequent}

\begin{sequent}
  \AxiomC{$b = \bcode{sget-byte}\ r_d\ c'.f$}
  \AxiomC{$\typeof(c'.f) = \databyte$}
  \AxiomC{$\checkaccess(c.m, b)$}
  \TrinaryInfC{$Q' = (\registers[r_d \mapsto \typeof(c'.f)], \rvoid)$}
\end{sequent}

\begin{sequent}
  \AxiomC{$b = \bcode{sget-char}\ r_d\ c'.f$}
  \AxiomC{$\typeof(c'.f) = \datachar$}
  \AxiomC{$\checkaccess(c.m, b)$}
  \TrinaryInfC{$Q' = (\registers[r_d \mapsto \typeof(c'.f)], \rvoid)$}
\end{sequent}

\begin{sequent}
  \AxiomC{$b = \bcode{sget-object}\ r_d\ c'.f$}
  \AxiomC{$\typeof(c'.f) \subtype \refobject{\jobject}$}
  \AxiomC{$\checkaccess(c.m, b)$}
  \TrinaryInfC{$Q' = (\registers[r_d \mapsto \typeof(c'.f)], \rvoid)$}
\end{sequent}

\begin{sequent}
  \AxiomC{$b = \bcode{sget-short}\ r_d\ c'.f$}
  \AxiomC{$\typeof(c'.f) = \datashort$}
  \AxiomC{$\checkaccess(c.m, b)$}
  \TrinaryInfC{$Q' = (\registers[r_d \mapsto \typeof(c'.f)], \rvoid)$}
\end{sequent}

\begin{sequent}
  \AxiomC{$b = \bcode{sget-wide}\ r_d\ c'.f$}
  \AxiomC{$\typeof(c'.f) \subtype \joinsq$}
  \AxiomC{$\checkaccess(c.m, b)$}
  \TrinaryInfC{$Q' = (\registers[r_d \mapsto \typeof(c'.f), r_d + 1 \mapsto \typeof(c'.f)], \rvoid)$}
\end{sequent}

\begin{sequent}
  \AxiomC{$b = \bcode{sput}\ r_s\ c'.f$}
  \AxiomC{$\registers(r_s) \subtype \typeof(c'.f) \subtype \jointd$}
  \AxiomC{$\checkaccess(c.m, b)$}
  \TrinaryInfC{$Q' = (\registers, \rvoid)$}
\end{sequent}

\begin{sequent}
  \AxiomC{$b = \bcode{sput-boolean}\ r_s\ c'.f$}
  \AxiomC{$\typeof(c'.f) = \databool$ \defaultHypSeparation $\registers(r_s) \subtype \typeof(c'.f)$}
  \AxiomC{$\checkaccess(c.m, b)$}
  \TrinaryInfC{$Q' = (\registers, \rvoid)$}
\end{sequent}

\begin{sequent}
  \AxiomC{$b = \bcode{sput-byte}\ r_s\ c'.f$}
  \AxiomC{$\typeof(c'.f) = \databyte$ \defaultHypSeparation $\registers(r_s) \subtype \typeof(c'.f)$}
  \AxiomC{$\checkaccess(c.m, b)$}
  \TrinaryInfC{$Q' = (\registers, \rvoid)$}
\end{sequent}

\begin{sequent}
  \AxiomC{$b = \bcode{sput-char}\ r_s\ c'.f$}
  \AxiomC{$\typeof(c'.f) = \datachar$ \defaultHypSeparation $\registers(r_s) \subtype \typeof(c'.f)$}
  \AxiomC{$\checkaccess(c.m, b)$}
  \TrinaryInfC{$Q' = (\registers, \rvoid)$}
\end{sequent}

\begin{sequent}
  \AxiomC{$b = \bcode{sput-object}\ r_s\ c'.f$}
  \AxiomC{$\registers(r_s) \subtype \typeof(c'.f) \subtype \refobject{\jobject}$}
  \AxiomC{$\checkaccess(c.m, b)$}
  \TrinaryInfC{$Q' = (\registers, \rvoid)$}
\end{sequent}

\begin{sequent}
  \AxiomC{$b = \bcode{sput-short}\ r_s\ c'.f$}
  \AxiomC{$\typeof(c'.f) = \datashort$ \defaultHypSeparation $\registers(r_s) \subtype \typeof(c'.f)$}
  \AxiomC{$\checkaccess(c.m, b)$}
  \TrinaryInfC{$Q' = (\registers, \rvoid)$}
\end{sequent}

\begin{sequent}
  \AxiomC{$b = \bcode{sput-wide}\ r_s\ c'.f$}
  \AxiomC{$\registers(r_s) \subtype \typeof(c'.f) \subtype \joinsq$ \defaultHypSeparation $\registers(r_s) = \registers(r_s + 1)$}
  \AxiomC{$\checkaccess(c.m, b)$}
  \TrinaryInfC{$Q' = (\registers, \rvoid)$}
\end{sequent}



\begin{sequent}
  \AxiomC{$b = \bcode{throw}\ r_e$}
  \AxiomC{$\registers(r_e) \subtype \refobject{\jexception}$}
  \BinaryInfC{$Q' = (\registers, \registers(r_e))$}
\end{sequent}

}



\section{TODO}
\begin{itemize}
\item initialisation corrected des objets et appel du constructeur immédiatement après le \bcode{new-instance}
  
\item parsing/décodage des instructions
  \begin{itemize}
  \item vérification des paramètres passés (\bcode{const}, \bcode{goto}, \bcode{filled*}, \dots)
  \item vérification des blocs monitorés
  \item pas de saut dans des trucs qui sont pas des instructions (apres parsing linéaire du bytecode)
  \end{itemize}


\end{itemize}






\end{document}

